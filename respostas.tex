\includegraphics[scale=0.5]{images/olho.png}
\vspace{150mm}
\fontencoding{T1}\fontfamily{cmss}\fontseries{bx}\fontshape{sl}


\textbf{Como é que avaliou o alinhamento da marcação do zero da régua com os dedos do seu colega? }

\fontencoding{T1}\fontfamily{cmss}\fontseries{sl}

   Com o objetivo de obter dados precisos, a marcação do zero da régua deve estar alinhada com o ponto onde os dedos do elemento do grupo estão posicionados antes da queda. Caso o zero não esteja bem alinhado, a distância medida será incorreta, afetando o tempo de reação e resultando em medidas pouco fiáveis. Assim, minimizamos a ocorrência de erros experimentais ao garantir que, em todas as medições, a posição inicial dos dedos em relação à régua seja sempre a mesma.  Ao garantir que os dedos no início da queda se encontram alinhados com o zero da régua, o cálculo do tempo de reação será mais preciso.                          Para garantir isto, tentamos encontrar uma forma de manter a mão estável e uma forma de alinhar os dedos de forma precisa, constante e eficiente. Para garantir a estabilidade da mão, apoiamos o antebraço numa superfície plana. Quanto a alinhar os dedos com o zero da régua, utilizamos uma outra superfície plana. Começamos por alinhar a régua perpendicularmente a ela e, então, posicionar os dedos.
\\
\fontencoding{T1}\fontfamily{cmss}\fontseries{bx}\fontshape{sl}

\textbf{Comente a experiência, a sua execução e resultados, e compare os tempos de reação dos elementos do grupo. }

\fontencoding{T1}\fontfamily{cmss}\fontseries{sl}
Ao executar a experiência, procuramos utilizar técnicas que nos permitem diminuir o valor da incerteza. Por exemplo, tentamos assegurar-nos que não havia movimentos do braço no decorrer da experiência. Para isso apoiamos o antebraço numa superfície plana. Para além disso, tentamos alinhar os dedos com o zero da régua utilizando sempre a mesma técnica.  

Todos os elementos apresentaram valores diferentes no tempo de reação. O elemento 2 foi o que apresentou o menor tempo de reação, enquanto que o elemento 3 foi o que teve um maior tempo. Para além disso, a incerteza associada à medição dos vários elementos é também diferente. Isto deve se ao facto de os elementos terem dedos de tamanho diferente; de cada elemento ter utilizado réguas diferentes; não foi a mesma pessoa a largar a régua e não foi a mesma pessoa a ler os valores na régua. 
\\
\fontencoding{T1}\fontfamily{cmss}\fontseries{bx}\fontshape{sl}

\textbf{Consegue estimar que importância terá para o resultado o afastamento entre os dedos enquanto se espera que a régua seja libertada? }

\fontencoding{T1}\fontfamily{cmss}\fontseries{sl}
O afastamento dos dedos afeta o resultado do tempo de reação. Quando os dedos estão, inicialmente, mais afastados, o valor do tempo de reação será maior. Isto acontece porque, quanto mais afastados estiverem os dedos, maior é a distância que os dedos precisam percorrer para agarrar a régua. Ou seja, a distância final dos dedos ao zero será superior ao que seria se os dedos estivessem mais próximos, como a um centímetro de distância.  

Portanto, utilizando o mesmo raciocínio, quanto mais próximos estiverem os dedos menor será o tempo de reação. 

É, por tanto, muito importante manter constante a distância entre os dedos à régua. Essa distância não deve ser muito grande, nem muito pequena. Neste caso, tentamos manter uma distância de, aproximadamente, um centímetro. 


