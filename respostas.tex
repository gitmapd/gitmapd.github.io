\vfill
\fontencoding{T1}\fontfamily{cmss}\fontseries{bx}\fontshape{sl}


\textbf{Como é que avaliou o alinhamento da marcação do zero da régua com os dedos do seu colega?}

\fontencoding{T1}\fontfamily{cmss}\fontseries{sl}

- Considerando a hipótese de os dedos estarem muito afastados, o movimento necessário para fechar os dedos e agarrar a régua será superior em comparação aquando os dedos estivessem a uma largura de 1 cm. Consequentemente, irá ocorrer um atraso no tempo de captura da régua, ou seja, iria afetar a posição em relação ao zero da régua. Deste modo o tempo de reação seria maior.
- Por outro lado, se os dedos estiverem muito próximos, existe uma possibilidade o tempo de reação ser inferior ao esperado, uma vez que a distância percorrida pela régua desde o zero até ao instante em que se agarra a mesma seria menor. 
- Deste modo, é necessário que para as várias medidas o afastamento entre os dedos seja constante a fim de evitar valores de medidas muito dispersos.
\\
\fontencoding{T1}\fontfamily{cmss}\fontseries{bx}\fontshape{sl}

\textbf{Explique o porquê da sua escolha. Consegue estimar que importância terá para o resultado o afastamento entre os dedos enquanto se espera que a régua seja libertada? }

\fontencoding{T1}\fontfamily{cmss}\fontseries{sl}
- Com o objetivo de obter dados precisos, a marcação do zero da régua deve estar alinhada com o ponto onde os dedos do elemento do grupo estão posicionados antes da queda. Caso o zero não esteja bem alinhado, a distância medida será incorreta, afetando o tempo de reação e resultando em medidas pouco fiáveis. Deste modo, minimizamos a ocorrência de erros experimentais através da garantia de que para todas as medições efetuadas a posição inicial dos dedos em relação à régua é igual. Ao garantir que os dedos no início da queda se encontram alinhados com o zero da régua, o cálculo do tempo de reação irá ser mais preciso. 

